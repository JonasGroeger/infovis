\documentclass[a4paper,12pt,ngerman]{scrartcl}

% Language
\usepackage{polyglossia}
\setmainlanguage{german}

% Grafiken
\usepackage{graphicx}

% Make \today print in format "NN<st|nd|rd|th> MM YYYY"
\usepackage{isodate}
\origdate

\linespread{1.15}

% Titel anders formatieren mit: Command, Format, Label, Sep, Before-Code
\usepackage{titlesec}
\titleformat{\section}{\sffamily\Large\bfseries}{}{0pt}{}
\titleformat{\subsection}{\sffamily\large\bfseries}{}{0pt}{}

% Own pagestyle (header and footer)
\usepackage{fancyhdr}
\fancyhf{} % Clear header and footer content
\fancyhead[L]{{\small \textsf{WS 14/15, Informationsvisualisierung}}}
\fancyhead[C]{{\small \textsf{Übung 5}}}
\fancyhead[R]{{\small \textsf{\today}}}
\fancyfoot[C]{\thepage}

% Text in quotes
% Usage: \enquote{To be or not to be.}
\usepackage{csquotes}

% Subliminal refinements towards typographical perfection
\usepackage{microtype}

% Footnotes in section
\usepackage[stable]{footmisc}

\usepackage{hyperref}

% Better refs with \cref{}
\usepackage[capitalize,noabbrev]{cleveref}

% Borders
\usepackage[paper=a4paper,left=20mm,right=20mm,top=30mm,bottom=30mm]{geometry}

\usepackage{placeins}

\begin{document}
\pagestyle{fancy} % Activate own pagestyle

\section{Aufgabe 5.1 | Schlagwortwolke}
Die in \Cref{fig:schlagwortwolke} dargestellte Schlagwortwolke ist mit dem Tool \href{https://tagul.com/}{Tagul} erstellt.

\begin{figure}[ht]
    \centering
    \includegraphics[width=0.85\textwidth]{includes/schlagwortwolke}
    \caption{Schlagwortwolke zum Thema Weihnachten}
    \label{fig:schlagwortwolke}
\end{figure}

\section{Aufgabe 5.2 | DNA-Sequenzen}
Die DNA Sequenzen sind im Ordner dna zu finden, da sie zu unförmig für das Einbinden in ein PDF sind. Die Sequenz ATGCCCCGTATTAGACGATAGACHTTAGATTAGATAG hatte einen Fehler mit Symbol H. Dieser wurde durch ein G ersetzt.
\end{document}

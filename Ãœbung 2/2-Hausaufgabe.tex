\documentclass[a4paper,12pt,ngerman]{scrartcl}

% Language
\usepackage{polyglossia}
\setmainlanguage{german}

% Grafiken
\usepackage{graphicx}

% Make \today print in format "NN<st|nd|rd|th> MM YYYY"
\usepackage{isodate}
\origdate

\linespread{1.15}

% Titel anders formatieren mit: Command, Format, Label, Sep, Before-Code
\usepackage{titlesec}
\titleformat{\section}{\sffamily\Large\bfseries}{}{0pt}{}
\titleformat{\subsection}{\sffamily\large\bfseries}{}{0pt}{}

% Own pagestyle (header and footer)
\usepackage{fancyhdr}
\fancyhf{} % Clear header and footer content
\fancyhead[L]{{\small \textsf{WS 14/15, Informationsvisualisierung}}}
\fancyhead[C]{{\small \textsf{Übung 2}}}
\fancyhead[R]{{\small \textsf{\today}}}
\fancyfoot[C]{\thepage}

% Text in quotes
% Usage: \enquote{To be or not to be.}
\usepackage{csquotes}

% Blindtext
\usepackage{blindtext}

% Subliminal refinements towards typographical perfection
\usepackage{microtype}

% Better refs with \cref{}
\usepackage[capitalize,noabbrev]{cleveref}

% Borders
\usepackage[paper=a4paper,left=20mm,right=20mm,top=30mm,bottom=30mm]{geometry}

\begin{document}
\pagestyle{fancy} % Activate own pagestyle

\section{Aufgabe 2.1 | Multivariate Daten}
In diesem Abschnitt werden Vor- und Nachteile sowie die Qualifikation und Disqualifikation verschiedener Daten für einige Visualisierungstechniken (i.e. Parallel Coordinates, Scatter Plots, Parallel Sets, Star Plot) beschrieben.

\subsection*{Parallel Coordinates}
Die Visualisierungstechnik Parallel Coordinates verwendet pro Dimension je eine eigene vertikale Linie. Mehrere dieser sind dann parallel mit einigem Abstand anzuordnen. Konkrete Datensätze sind dann quer zu den vertikelen Linien verlaufende Linien. Diese schneiden sich an der konkreten Wertausprägung mit den vertikalen Linien.

\begin{enumerate}
	\item \textbf{Vorteile:}
	\item \textbf{Nachteile:}
	\item \textbf{Daten die diese Methode qualifizieren:}
	\item \textbf{Daten die diese Methode disqualifizieren:}
\end{enumerate}

\subsection*{Scatter Plot}
\blindtext
\begin{enumerate}
	\item \textbf{Vorteile:} Schnelle Übersicht und Strukturfindung bzw. Zusammenhänge
	\item \textbf{Nachteile:}
	\item \textbf{Daten die diese Methode qualifizieren:}
	\item \textbf{Daten die diese Methode disqualifizieren:} Bei einer großen Anzahl von Dimensionen wird die Matrix schnell sehr groß.
\end{enumerate}

\subsection*{Parallel Sets}
\blindtext
\begin{enumerate}
	\item \textbf{Vorteile:}
	\item \textbf{Nachteile:}
	\item \textbf{Daten die diese Methode qualifizieren:} Nominale
	\item \textbf{Daten die diese Methode disqualifizieren:}
\end{enumerate}

\subsection*{Star Plot}
\blindtext
\begin{enumerate}
	\item \textbf{Vorteile:}
	\item \textbf{Nachteile:}
	\item \textbf{Daten die diese Methode qualifizieren:}
	\item \textbf{Daten die diese Methode disqualifizieren:}
\end{enumerate}

\section{Aufgabe 2.2 | RadViz}
\blindtext

\section{Aufgabe 2.3 | Visualisierung}
\blindtext

\end{document}

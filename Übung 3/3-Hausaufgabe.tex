\documentclass[a4paper,12pt,ngerman]{scrartcl}

% Language
\usepackage{polyglossia}
\setmainlanguage{german}

% Grafiken
\usepackage{graphicx}

% Make \today print in format "NN<st|nd|rd|th> MM YYYY"
\usepackage{isodate}
\origdate

\linespread{1.15}

% Titel anders formatieren mit: Command, Format, Label, Sep, Before-Code
\usepackage{titlesec}
\titleformat{\section}{\sffamily\Large\bfseries}{}{0pt}{}
\titleformat{\subsection}{\sffamily\large\bfseries}{}{0pt}{}

% Own pagestyle (header and footer)
\usepackage{fancyhdr}
\fancyhf{} % Clear header and footer content
\fancyhead[L]{{\small \textsf{WS 14/15, Informationsvisualisierung}}}
\fancyhead[C]{{\small \textsf{Übung 3}}}
\fancyhead[R]{{\small \textsf{\today}}}
\fancyfoot[C]{\thepage}

% Text in quotes
% Usage: \enquote{To be or not to be.}
\usepackage{csquotes}

% Subliminal refinements towards typographical perfection
\usepackage{microtype}

% Footnotes in section
\usepackage[stable]{footmisc}

% Better refs with \cref{}
\usepackage[capitalize,noabbrev]{cleveref}

% Borders
\usepackage[paper=a4paper,left=20mm,right=20mm,top=30mm,bottom=30mm]{geometry}

\usepackage{longtable}

\begin{document}
\pagestyle{fancy} % Activate own pagestyle

\section{Aufgabe 3.1 | Visuelle Eigenschaften von Graphen}
\subsection*{XXX}

\section{Aufgabe 3.2 | Visual Clutter}
\subsection*{Kamada-Kawai}
\begin{itemize}
\item Rahmenbedingungen
\begin{itemize}
\item ungerichtete, gewichtete Graphen
\item gerade Kanten, keine Beschränkung bei Positionierung der Knoten
\item zusammenhängender Graph (sonst: Zusammenhangskomponenten einzeln zeichnen)
\end{itemize}
\item Kriterien
\begin{itemize}
\item Anzahl Kantenkreuzungen reduzieren
\item gleichmäßige Verteilung von Knoten und Kanten (wichtiger für menschliches Verständnis)
\end{itemize}
\item Idee
\begin{itemize}
\item Stahlringe werden durch Federn zusammengehalten
\item Ziel: Minimierung der Energie dieses Systems
\item ideale Distanz zwischen zwei Knoten proportional zur Länge eines kürzesten Pfades zwischen ihnen
\end{itemize}
\item Algorithmus
\begin{itemize}
\item Berechnung der Distanz $d_{i,j}$  (Shortest Path Algorithm - Floyd)
\item Berechnung der Länge $l_{i,j}$ (aus Distanz und optimaler Länge)
\item Berechnung der Federstärke $k_{i,j}$ (aus Distanz)
\item Bestimmung der initialen Knotenpositionen
\item Schrittweise Minimierung der Energie: Verschieben eines Knotens in stabile Position
\end{itemize}

Laufzeit: O($|V|^3 + T * |V|$), $T =$ Anzahl der Iterationen
\item Besonderheiten
\begin{itemize}
\item symmetrische Graphen werden auch symmetrisch angeordnet
\item isomorphe Graphen werden auf dieselbe Art und Weise dargestellt (evtl. verschoben, gedreht oder gespiegelt)
\item in gewichteten Graphen: Distanz = Summe der Gewichte
\end{itemize}
\end{itemize}
\subsection*{Fruchtermann-Reingold}
\subsection*{Gemeinsamkeiten}
\subsection*{Unterschiede}

\section{Aufgabe 3.3 | Graphen übersichtlicher gestalten}

\subsection*{Einfärben von Knoten oder Kanten}
\subsection*{Gruppieren von Knoten und Kanten}
\subsection*{Verschiedene Knotendarstellungen}

\section{Aufgabe 3.4 | Visualisierung mit Gephi}

%\begin{figure}[ht]
%    \centering
%    \includegraphics[height=8cm]{includes/gephi1}
%    \caption{Grafik direkt nach Import der Daten}
%    \label{fig:gephi1}
%\end{figure}
%
%\begin{figure}[ht]
%    \centering
%    \includegraphics[height=8cm]{includes/gephi2}
%    \caption{Grafik direkt nach Import der Daten}
%    \label{fig:gephi2}
%\end{figure}

Für die Visualisierungen mit Gephi wird hier der Datensatz der Facebook-Freundschaften der empfohlenen Seite verwendet, der aus 10 Netzwerken zusammengesetzt wurde \footnote{Größere Datensätze waren im CIP-Pool leider nicht zu importieren. Selbst mit diesem relativ kleinen Datensatz gab es immer wieder Probleme mit dem begrenzten Speicherplatz. Auf meinem Windows-PC wollte Gephi leider nicht starten.}.

Gephi beginnt nach dem Import des Datensatzes mit einem zufälligen Layout, das die Knoten innerhalb einer rechteckigen Fläche anordnet, siehe \cref{fig:start}. In dieser Darstellung sind bei Datensätzen solcher Größe vermutlich selten Aussagen über die Daten zu treffen.

\begin{figure}[ht]
    \centering
    \includegraphics[height=8cm]{includes/snapshot_start}
    \caption{Grafik direkt nach Import der Daten}
    \label{fig:start}
\end{figure}

In erster Linie kann die Visualisierung durch ein Layout verbessert werden, das nach den Beziehungen zwischen den Daten ausgerichtet ist. Die Erfahrung zeigt, dass oft eine Kombination verschiedener Algorithmen ästhetischere Abbildungen ergibt, aus denen bereits erste Erkenntnisse gewonnen werden können. \Cref{fig:einfarbig} zeigt ein solches Layout.

\begin{figure}[ht]
    \centering
    \includegraphics[height=8cm]{includes/snapshot_einfarbig}
    \caption{Grafik nach Anwendung von Layout-Algorithmen}
    \label{fig:einfarbig}
\end{figure}

Zusätzlich können Knoten und Kanten noch eingefärbt werden. Zu den zahlreichen Möglichkeiten gehören unter anderem (Ein-/Ausgangs-)Grad, Kantengewicht oder Zentralitätsmaße. Am nützlichsten hat sich für diesen Datensatz die Färbung der Knoten nach \emph{Modularity} erwiesen, wie in \cref{fig:bunt} zu sehen.

\begin{figure}[ht]
    \centering
    \includegraphics[height=8cm]{includes/snapshot_bunt}
    \caption{Grafik nach Anwendung von Layout-Algorithmen und Einfärben der Knoten}
    \label{fig:bunt}
\end{figure}
\end{document}
